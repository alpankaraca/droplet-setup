\documentclass[12pt, a4paper]{article}
\usepackage[utf8]{inputenc}
\usepackage{csdnws}
%\coursename{CMN 167}
%\week{1}

\begin{document}
%\maketitle

\section*{CMPE 283 Worksheet 09}
\section*{Project 09} 
Deadline 0900 Monday, December 8


\section*{Classwork}

This week we will extend the calculator game to use a database.

\begin{enumerate}
 \item Add the facility to have users create accounts with passwords
 \item Keep scores in the database.
 \item Display a top ten scorers table.
\end{enumerate}

You should be able to do many of the tasks required from this course from a tablet or even a smart phone. However, the easiest and most productive way to do the practical work in this course will be from a personal computer with Linux installed. Solving the problems you will encounter with \emph{any} method of accessing your server is \emph{part of the course}.

\subsection*{Coaching}

You can gain extra points, possibly making up for previous missed projects or low exam grades by coaching other students. To gain extra points for this, you need to do this by \emph{posting in the on-line self help forum}. Even if you meet your fellow students face to face to help them get these projects done, you should write down the exchanges between you, preferably as they happen, and post them on-line. I will be awarding grades for helpful, clear posts, also for well formulated smart questions. 

\subsection*{Setting up Linux on your own computer}

See the instructions on previous worksheets.


\section*{Assignment}

\begin{enumerate}
\item Finish up what you started in class. Be prepared to show what you have done in the next lab class. 
\item Write up an account of what you did and what you learned. What was wrong or missing in these instructions? Could you now explain what you have done so far to a friend?
\item Post your account to online.bilgi.edu.tr along with a link to the appropriate domain on your droplet and its IP number.

\end{enumerate}

\copyright Chris Stephenson 2014

\end{document}
