\documentclass[12pt, a4paper]{article}
\usepackage[utf8]{inputenc}
\usepackage{csdnws}
%\coursename{CMN 167}
%\week{1}

\begin{document}
%\maketitle

\section*{CMPE 283 Worksheet 07}
\section*{Project 07} 
Deadline 0900 Monday, November 24


\section*{Classwork}
Set up a third domain on your droplet and test the php files from the lecture in this domain.

Now try extending the files so that you can implement a ``calculator game''. The calculator game presents a web form with a random calculation on it (you can start with an addition) and invites the user to try input the answer. If the answer is correct give the user a congratulation, and present a new random calculation. If the answer is wrong, ask the user to try again.

You will need at least three php files to do this, and you will need to use the online php manual at php.net to find out how to generate random numbers and also how to do arithmetic in php.

You should be able to do many of the tasks required from this course from a tablet or even a smart phone. However, the easiest and most productive way to do the practical work in this course will be from a personal computer with Linux installed. Solving the problems you will encounter with \emph{any} method of accessing your server is \emph{part of the course}.

\subsection*{Setting up Linux on your own computer}

See the instructions on previous workheets.


\section*{Assignment}

\begin{enumerate}
\item Finish up what you started in class. Be prepared to show what you have done in the next lab class. 
\item Extend your calculator game to maintain a score of right and wrong answers for each user. You will need to use additional (and possibly hidden) form fields to do this. Yo should be able to have several diffrent users playing the calculator game on your site at the same time without their scores getting muddled with one another. Think of and implement of other ways to extend the calculator game. 
\item Write up an account of what you did and what you learned. What was wrong or missing in these instructions? Could you now explain what you have done so far to a friend?
\item Post your account to online.bilgi.edu.tr along with a link to the appropriate domain on your droplet and its IP number.

\end{enumerate}

\copyright Chris Stephenson 2014

\end{document}
